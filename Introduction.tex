\section{Introduction}
At DTU, several student teams are working with building cars for different purposes. For example, the Ecocar is competing for driving the furthest possible on a liter of fuel, the Solar car is trying to drive on sunpower only, and Vermilion Racing is building an electric racecar to compete in Formula Student. The Vermilion racecar is the inspiration for this project, where the topic is to investigate if LoRa is a suitable technology for wireless telemetry in situations, reading sensors on a high speed moving vehicle in real time.

A concrete case is that the car in the summer is shipped to Italy, where the car is gonna race on the Riccardo Paletti di Varano `de Melegrari racetrack, near Parma. On the track, the distance between the car and the paddock where the team are during the race, is approximately 600 meters, and the biggest distance from one point to another on the track is just shy of one kilometer. The terrain on and around the track is mostly unobstructed.
